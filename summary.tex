\section{Future Work}
This section describes future work beyond the scope of this dissertation.

The immediate next research steps for the volumetric NED could be the real-time implementation of the proposed rendering pipeline. This is not a trivial improvement because of the large computation and communication demands that the system needs to address. However, addressing this large computation and communication demand should be possible with this NED because its components were originally designed for a low-latency AR display~\cite{Lincoln2016motion}. So, in addition to a real-time volumetric NED, future work could include developing a low-latency volumetric NED.

The ability of our varifocal occlusion-capable AR display to attenuate real-world light can also be used to depict consistent illumination in the AR scene and depict interesting effects such as shadows cast by virtual objects onto the real world and vice-versa, or to relight the real-world to match the virtual scene (such effects have been demonstrated in a projection-based lighting system~\cite{bimber2003consistent}).

Both of the presented display technologies can also emulate multiple other AR displays, e.g., the volumetric NED can also emulate previous varifocal NEDs and previous multifocal NEDs, and the varifocal occlusion-capable NED can also emulate fixed-focus occlusion displays and occlusion-incapable varifocal AR displays. Hence, this dissertation's displays could be used as test-beds to conduct perceptual experiments to understand the human visual system better and to come up with strategies for future NEDs.
