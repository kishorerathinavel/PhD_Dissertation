% citation style

% default: cite with (Name, year)
\renewcommand{\cite}{\citep}

% common abbreviations
\newcommand{\eg}{{\it e.g.}\xspace}
\newcommand{\ie}{{\it i.e.}\xspace}
\newcommand{\etc}{{\it etc.}\xspace}
\newcommand{\etal}{\emph{et~al}\mbox{.}\xspace}

\newcommand{\xth}{\ensuremath{^{\text{th}}}\xspace}
%\newcommand{\fst}{\ensuremath{^{\text{st}}}\xspace}



\newcommand{\vs}{{vs\mbox{.}}\xspace}

% common Math notation
\newcommand{\NAT}[0]{\mathbb{N}\xspace}
\newcommand{\fun}[1]{\mathit{#1}} % typeset as function name
\newcommand{\setsize}[1]{\left| #1 \right|}
\newcommand{\setdef}[2]{\left\{ #1 \ \left|\  #2\right.\right\}}
\newcommand{\dispsum}[0]{\displaystyle\sum}

\newcommand{\defeq}[0]{\triangleq}
\renewcommand{\mod}{\operatorname{mod}}

% time units
\newcommand{\mus}[0]{\ensuremath{\mu s}\xspace}
\newcommand{\us}[0]{\ensuremath{\mu s}}
\newcommand{\ms}[0]{\ensuremath{\fun{ms}}\xspace}

% algorithm names
\newcommand{\kwfont}[1]{\textsf{#1}\xspace} %\small
% variable name
\newcommand{\var}[1]{\ensuremath{{\fun{#1}}}\xspace} %\small

%http://hstuart.dk/2007/08/03/programming-latex-%E2%80%94-writing-commands/
\newcommand{\mkkw}[2]{
	\newcommand{#1}[0]{\kwfont{#2}}
}

% fancy symbols and functions
\newcommand{\Alg}[0]{{\mathcal A}}
\newcommand{\Test}[0]{{\mathcal T}}
\newcommand{\Mach}[0]{{\mathcal M}}

\newcommand{\usum}[0]{u_{\mathrm{sum}}}
\newcommand{\umax}[0]{u_{\mathrm{max}}}
\newcommand{\umin}[0]{u_{\mathrm{min}}}
\newcommand{\utop}[0]{u_{\mathrm{top}}}

\newcommand{\esum}[0]{e_{\mathrm{sum}}}
\newcommand{\emax}[0]{e_{\mathrm{max}}}
\newcommand{\emin}[0]{e_{\mathrm{min}}}
\newcommand{\etop}[0]{e_{\mathrm{top}}}

\newcommand{\dsum}[0]{\delta_{\mathrm{sum}}}
\newcommand{\dmax}[0]{\delta_{\mathrm{max}}}
\newcommand{\dmin}[0]{\delta_{\mathrm{min}}}
\newcommand{\dtop}[0]{\delta_{\mathrm{top}}}

\newcommand{\prio}[0]{\mathsf Y}
\newcommand{\eprio}[0]{\mathsf y}

\DeclareMathOperator*{\argmin}{arg\!\min}

% src code
\newcommand{\src}[1]{\textsf{\small #1}\xspace}

% Kishore
\newcommand{\numPlanes}{N_{\text{planes}}}

\usepackage{xcolor,colortbl}

\definecolor{Purple}{rgb}{0.5, 0., 0.5}
\definecolor{Green}{rgb}{0.0, 0.5, 0.0}
\definecolor{Blue}{rgb}{0.0, 0.0, 1.0}
\definecolor{Red}{rgb}{1.0, 0.0, 0.0}
\definecolor{DarkGreen}{rgb}{0.0, 0.5, 0.0}
\definecolor{Gray}{rgb}{0.5 0.5, 0.5}
\definecolor{Fuchsia}{rgb}{0.79,0.17,0.57}

\newcommand{\todo}[1]{\textcolor{Red}{[To do: #1]}}

\newenvironment{compact_enumerate}{
     \begin{enumerate}[leftmargin=0.4cm]
     \setlength{\itemsep}{1pt}
     \setlength{\parskip}{0pt}
     \setlength{\parsep}{0pt}
   }{\end{enumerate}}

\newenvironment{compact_itemize}{
  \begin{itemize}[leftmargin=0.4cm]
    \setlength{\itemsep}{1pt}
    \setlength{\parskip}{0pt}
    \setlength{\parsep}{0pt}
  }{\end{itemize}}

\usepackage{enumitem}
\newlist{todolist}{itemize}{6}
\setlist[todolist]{label=$\square$}
\setlist{nolistsep}

\newcommand{\done}{\rlap{$\square$}{\raisebox{2pt}{\large\hspace{1pt}\checkmark}}%
\hspace{-2.5pt}}

\newenvironment{compact_todolist}{
  \begin{todolist}[leftmargin=0.4cm]
    \setlength{\itemsep}{1pt}
    \setlength{\parskip}{0pt}
    \setlength{\parsep}{0pt}
  }{\end{todolist}}

\usepackage{fancyhdr}
\pagestyle{fancy}
\fancyfoot{}
\fancyfoot[L]{\thepage}
\fancyfoot[R]{\hyperlink{toc}{TOC}}
\fancyhead{}
\renewcommand{\headrulewidth}{0pt}
%\fancyhead[R]{\slshape \nouppercase{\leftmark}}

\usepackage{csquotes}
\SetBlockThreshold{2}
\newcommand\myblockquote[2]{%
  \blockquote{\hspace*{2em}\emph{`#1'}\hfill #2}\par}

% End of Kishore

% schedulers
\mkkw{\cfs}{CFS}

\mkkw{\edf}{EDF}
\mkkw{\edfwm}{EDF-WM}
\mkkw{\fp}{FP}
\mkkw{\fprm}{RM}
\mkkw{\fpdm}{DM}
\mkkw{\gedf}{G-EDF}
\mkkw{\gsnedf}{GSN-EDF}
\mkkw{\gfp}{G-FP}
\mkkw{\pedf}{P-EDF}
\mkkw{\pfp}{P-FP}
\mkkw{\cedf}{C-EDF}
\mkkw{\pssched}{PS}

\mkkw{\pfsched}{PF}
\mkkw{\pd}{PD}
\mkkw{\pds}{PD$^2$}
\mkkw{\cpds}{C-PD$^2$}

\mkkw{\jlfp}{JLFP}
\mkkw{\jldp}{JLDP}


% Plugins

\mkkw{\pfpgi}{P-FP-Rm}
\mkkw{\pfpdi}{P-FP-R1}
\mkkw{\pedfgi}{P-EDF-Rm}
\mkkw{\pedfdi}{P-EDF-R1}
\mkkw{\cedfiigi}{C2-EDF-Rm}
\mkkw{\cedfiidi}{C2-EDF-R1}
\mkkw{\cedfiiigi}{C6-EDF-Rm}
\mkkw{\cedfiiidi}{C6-EDF-R1}
\mkkw{\gedfgi}{G-EDF-Rm}
\mkkw{\gedfdi}{G-EDF-R1}

\mkkw{\cedfalldi}{C24-EDF-R1}

\mkkw{\capdsiigi}{C2-aPD$^2$-Rm}
\mkkw{\capdsiidi}{C2-aPD$^2$-R1}

\mkkw{\capdsiiigi}{C6-aPD$^2$-Rm}
\mkkw{\capdsiiidi}{C6-aPD$^2$-R1}

\mkkw{\gapdsgi}{G-aPD$^2$-Rm}
\mkkw{\gapdsdi}{G-aPD$^2$-R1}

\mkkw{\cspdsiigi}{C2-sPD$^2$-Rm}
\mkkw{\cspdsiidi}{C2-sPD$^2$-R1}

\mkkw{\cspdsiiigi}{C6-sPD$^2$-Rm}
\mkkw{\cspdsiiidi}{C6-sPD$^2$-R1}

\mkkw{\gspdsgi}{G-sPD$^2$-Rm}
\mkkw{\gspdsdi}{G-sPD$^2$-R1}


% POSIX

\mkkw{\schedfifo}{SCHED\_FIFO}
\mkkw{\schedrr}{SCHED\_RR}
\mkkw{\schedother}{SCHED\_OTHER}
\mkkw{\schedspor}{SCHED\_SPORADIC}
\mkkw{\prioprot}{PRIO\_PROTECT}
\mkkw{\scheddl}{SCHED\_DEADLINE}

% locking protocols
\mkkw{\npcs}{NCP}
\mkkw{\srp}{SRP}
\mkkw{\pcp}{PCP}
\mkkw{\msrp}{MSRP}
\mkkw{\dpcp}{DPCP}
\mkkw{\mpcp}{MPCP}
\mkkw{\mpcpvs}{MPCP-VS}
\mkkw{\fmlp}{FMLP}
\mkkw{\fmlpp}{FMLP$^{\mathrm{+}}$}
\mkkw{\npfmlpp}{NP-FMLP$^{\mathrm{+}}$}
\mkkw{\omlp}{OMLP}
\mkkw{\pip}{PIP}

% RW lock implementations
\mkkw{\pft}{PF-T}   % simple PF RW lock
\mkkw{\pfc}{PF-C}   % compact PF RW lock 
\mkkw{\pfq}{PF-Q}  % queue PF RW lock
\mkkw{\rwlin}{LX-RW}
\mkkw{\tft}{TF-T}
\mkkw{\tfq}{TF-Q}
\mkkw{\mtxt}{MX-T}
\mkkw{\mtxq}{MX-Q}

% locking protocol details
\newcommand{\FQ}[1]{FQ$_{#1}$}
\newcommand{\PQ}[1]{PQ$_{#1}$}

\newcommand{\WQ}[1]{WQ$_{#1}$}
\newcommand{\RQA}[1]{CQ$_{#1}$}
\newcommand{\RQP}[1]{DQ$_{#1}$}
\newcommand{\RQ}[2]{RQ$_{#1}^{#2}$}
\newcommand{\RQi}[1]{\RQ{#1}{1}}
\newcommand{\RQii}[1]{\RQ{#1}{2}}

\newcommand{\KQ}[1]{KQ$_{#1}$}
\newcommand{\KQq}[0]{\KQ{q}\xspace}
\newcommand{\RS}[1]{RS$_{#1}$}
\newcommand{\RSq}[0]{\RS{q}\xspace}

\newcommand{\BQ}[1]{BQ$_{#1}$}

\newcommand{\mc}[0]{\frac{m}{c}}
\newcommand{\nc}[0]{\frac{n}{c}}
\newcommand{\lmax}{L^{\fun{max}}}
\newcommand{\kmin}{k^{\fun{min}}}

% references
\newcommand{\chref}[1]{Chapter~\ref{ch:#1}\xspace}
\newcommand{\chrefs}[2]{Chapters~\ref{ch:#1} and~\ref{ch:#2}\xspace}
\newcommand{\secref}[1]{Section~\ref{sec:#1}\xspace}
\newcommand{\figref}[1]{Figure~\ref{fig:#1}\xspace}
\newcommand{\figrefi}[2]{Figure~\ref{fig:#1}(#2)\xspace}
\newcommand{\tabref}[1]{Table~\ref{tab:#1}\xspace}
\newcommand{\lemref}[1]{Lemma~\ref{lem:#1}\xspace}
\newcommand{\thmref}[1]{Theorem~\ref{thm:#1}\xspace}
\newcommand{\defref}[1]{Definition~\ref{def:#1}\xspace}
\newcommand{\exref}[1]{Example~\ref{ex:#1}\xspace}
\newcommand{\equref}[1]{Equation~(\ref{eq:#1})\xspace}
\newcommand{\inequref}[1]{Inequality~(\ref{eq:#1})\xspace}
\newcommand{\lstref}[1]{Listing~\ref{lst:#1}\xspace}
\newcommand{\pref}[1]{page~\pageref{p:#1}\xspace}
% citations

% resource notation
\newcommand{\res}[0]{\ell}
\newcommand{\req}[0]{\mathcal{R}}
\newcommand{\wreq}[0]{\mathcal{R}^{\fun{W}}}
\newcommand{\rreq}[0]{\mathcal{R}^{\fun{R}}}
\newcommand{\rlen}[0]{{\mathcal L}}
\newcommand{\bspin}[0]{s}

% locking proofs
\newcommand{\tauseq}[0]{\tau^{\fun{seq}}}

% overheads
\newcommand{\evlat}[0]{\Delta^{\fun{ev}}}
\newcommand{\ipilat}[0]{\Delta^{\fun{ipi}}}
\newcommand{\reloh}[0]{\Delta^{\fun{rel}}}
\newcommand{\schedoh}[0]{\Delta^{\fun{sch}}}
\newcommand{\cxsoh}[0]{\Delta^{\fun{cxs}}}
\newcommand{\cpmdoh}[0]{\Delta^{\fun{cpd}}}
\newcommand{\cidoh}[0]{\Delta^{\fun{cid}}}
\newcommand{\tickoh}[0]{\Delta^{\fun{tck}}}

\newcommand{\inoh}[0]{\Delta^{\fun{in}}}
\newcommand{\outoh}[0]{\Delta^{\fun{out}}}

\newcommand{\sysinoh}[0]{\Delta^{\fun{sci}}}
\newcommand{\sysoutoh}[0]{\Delta^{\fun{sco}}}


\newcommand{\numres}[0]{{n_r}}

% complicated names
\newcommand{\litmus}{LITMUS$^{\mathrm{RT}}$\xspace}
\newcommand{\aquo}{AQuoSA\xspace}
\newcommand{\prt}{PREEMPT\_RT\xspace}
\newcommand{\livlin}{L$^4$Linux\xspace}

% parameters

\newcommand{\pacc}[0]{\var{pacc}}
\newcommand{\wratio}[0]{\var{wratio}}

% special footnotes

% from http://help-csli.stanford.edu/tex/latex-footnotes.shtml
\long\def\symbolfootnote[#1]#2{\begingroup%
\def\thefootnote{\fnsymbol{footnote}}\footnote[#1]{#2}\endgroup}

% Theorems, etc.

\newtheoremstyle{mylemthm}% hnamei 
        {6pt}% hSpace abovei 
        {3pt}% hSpace belowi 
        {\slshape}% hBody fonti 
        {}% hIndent amounti1
        {\bfseries}% hTheorem head fonti 
        {.}% hPunctuation after theorem headi 
        {.5em}% hSpace after theorem headi2
        {}% hTheorem head spec (can be left empty, meaning `normal')i

\theoremstyle{mylemthm}

\newtheorem{theorem}{Theorem}[chapter]
\newtheorem{lemma}{Lemma}[chapter]

%\theoremstyle{definition}

\newtheoremstyle{mydef}% hnamei 
        {3pt}% hSpace abovei 
        {3pt}% hSpace belowi 
        {\normalfont}% hBody fonti 
        {}% hIndent amounti1
        {\bfseries}% hTheorem head fonti 
        {.}% hPunctuation after theorem headi 
        {.5em}% hSpace after theorem headi2
        {\thmname{#1} \thmnumber{#2}\thmnote{#3}}% hTheorem head spec (can be left empty, meaning `normal')i

\theoremstyle{mydef}


%% Flush words right at end of paragraph.
%% From: http://tex.stackexchange.com/questions/16330/hfill-after-linebreak
\newcommand\rightparend[1]{{%
      \unskip\nobreak\hfil\penalty50
      \hskip2em\hbox{}\nobreak\hfil\textbf{#1}%
      \parfillskip=0pt \finalhyphendemerits=0 \par}}


\newtheorem{definition}{Definition}[chapter]
\newtheorem{xxexample}{Example}[chapter]

%% "inherent" from xxexample, but place box at the end of example.
\newenvironment{example}{
\begin{xxexample}
}{
\rightparend{$\Diamond$}
\end{xxexample}
}
% \qed   \sqbullet \blackdiamond \vartriangleleft
