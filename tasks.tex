\chapter{Defense}
\section{Henry's advice to improve defense slides}
\begin{itemize}
\item Do not exceed 50 minutes. Think of how you'll gratiously end this.
\item Have a running header: Introduction, Volumetric, \dots
\item Explain AR vs VR from the title itself. Explain the real world scene, AR content, different distances. And mention that this is different from AR because you can also see the real world. \myverbatim{And let me just say that there is an entire volume here and when you focus at different distances, it will be in sharp focus. Full volume being displayed is closer to how things are in the real world.}
\item Make a new teaser with a 2 by 2 grid for the volumetric as well as varifocal occlusion results
\item Title: Highlight title and affiliation for non-UNC committee members
\item Example image for \myverbatim{next generation computing platform}: choose HoloLens images with many virtual displays
\item Choose a different tele-collaboration image
\item Label commercial AR devices
\item Use Augmented Reality ``Systems'' and not ``Displays''
\item Limitations of current generation AR displays - make minislides
\item Insert slide for limitations of current AR displays for depth cues in detail
\item Contributions of ``This'' (not ``My'') Thesis
\item Small images for Contributions
\item Does not cover chromablur and intra-pupillary occlusions but these are also less important than cues such as accommodation, defocus-blur, and occlusion
\item \myverbatim{Perceptually realistic and continuous monocular depth cues over a large depth range}: Mention contributions of each display separately. For volumetric, mention that the objects are displayed ``simultaneously'' at all depths. For varifocal, mention that the depth plane is dynamically adjustable.
\item Explain accommodation clearly. Explain occlusion clearly.
\item Here's what I've done: 280 depth planes possible by stripping out all the usual display interface and designing every aspect of the GPU to photon pipeline.
\item Both of these are enabled by new hardware (focus-tunable lenses) but together with new optical designs and algorithms
\item Acknowledge that MagicLeap has two depth planes. Check where MagicLeap puts its two depth planes. 
\item Put about Photonics West submission - credit Hanpeng.
\item Put overview of what to expect in each subtitle slide
\item About real-time system
    \begin{itemize}
    \item Towards Real-Time System: Initial Experiment
    \item Experiment to just get something running
    \item Layout of FPGA is difficult and here's what is achievable with the current FPGA Layout
    \item Why only two depth planes?
    \item Need a pipeline diagram
    \item Everything is happening in the GPU
    \end{itemize}
\item Occlusion as the most important depth cue
\item Previous work:
    \begin{itemize}
    \item AR displays - usually no occlusion
    \item Video see through - many limitations, e.g. Canon MREAL
    \end{itemize}
\item Overview of how we achieved varifocal occlusion: focus-tunable lens + maths
\end{itemize}

\section{Defense Tasks}
\begin{todolist}
\item Write the speech
\item Make introduction slides
\item Change the 300 label to 280
\item Make sure the proposed candidate diagrams are aligned properly
\end{todolist}

\chapter{Dissertation}

\section{Tasks}
\begin{todolist}
\item Write down outline of current dissertation draft
\item Write down what things are missing
\item Write down outline for next dissertation draft
\item Related Works: Look at some other dissertations
    \begin{compact_todolist}
    \item Pull in info on foveation from NVIDIA's recent paper
    \end{compact_todolist}
\item Volumetric Display:
    \begin{compact_todolist}
    \item Is there a way to prove that your display is better than contemporary displays
    \end{compact_todolist}
\item Write down these algorithms for color-adaptive decomposition:
    \begin{todolist}
    \item adaptive\_color\_decomposition
    \item adaptive\_color\_decomposition\_all\_channels
    \item heuristic\_adaptive\_decomposition
    \item highest\_energy\_channel\_decomposition
    \item ColorDC\_edit3
    \end{todolist}
\item Things of note to write down for above algorithms:
    \begin{todolist}
    \item Non-fixed pipeline algorithms tend to use more binary voxels, but tend to concentrate the energies in a narrow region. Maybe a threshold can be used to stop carrying over the residual energy. 
    \item Main gains are in terms of variance of binary voxel depths - which is an indication of depth blur. Is there a way to quantify this in terms of diopters?
    \item Single channel projected gradient algorithm is not necessary because the initialization is good enough. Further optimization never happens.
    \item Mixed-primary projected gradient algorihtm sucks. Main reason is that there is really not that much correspondence between the three color channels - this can be shown by counting the number of times the heuristic\_adaptive\_decomposition algorithm picks a mixed-primary combination, which is almost never. The bunny image is not a good example because it is not composed of mixed primaries. What about something with more complex colors - how do adaptive\_color\_decomposition\_all\_channels and heuristic\_adaptive\_decomposition do?
    \item This method does not penalize errors in residual that are carried over for several iterations, e.g. a residual at frame 10 may be carried over up to frame 100 because there is no mechanism in this approach to penalize discrepancy in depth.
    \item Re-write this in terms of error diffusion: This method has the effect that imagery towards the nearer end of the volume are properly represented and have low errors but imagery towards the farther end of the volume are not represented properly and have high errors. This is because imagery towards the farther end do not have sufficient number of frames to be represented. Recommended fix: What has been implemented so far is a forwards decomposition where the subvolume iterates from 1 to 280. What is needed is a backwards decomposition too, but for the backwards decomposition, the following needs to be changed: The initialization should be the $\alpha$ values and $B$ calculated in the forwards decomposition. Should the residual be a carry over from the forwards decomposition or should we re-initialize the residual to a zero image?
    \end{todolist}
\item Include pictures to compare the four algorithms
\item Make blockdiagram of the real-time system. Share it with the PhD committee
\end{todolist}

\section{Outline}
\begin{compact_itemize}
\item Abstract
    \begin{compact_itemize}
    \item Potential of being next generation computing platform
    \item Potential of seamlessly combine physical and digital content
    \item Needs perceptually realistic depth cues
    \item Limitation of today's commercial AR displays
    \item Dissertation's uses computational display's approach
    \item Approach for providing realistic accommodation
    \item Approach for providing realistic depth-dependent occlusion
    \item Overview of dissertation's contributions
    \end{compact_itemize}
\item Introduction
    \begin{compact_itemize}
    \item What are Augmented Reality Displays?
        \begin{compact_todolist}
        \item Currently, 2D monitors and mobile phones are our primary user interface with a computer.
        \item This confines our interaction with comptuers, because while we operate in a rich 3D world with fantastic vision capabilities, we're limited to using 2D screens of limited size. 
        \item These displays simply don't provide a sense of immersion or presence. Current computers can let you look at and edit virtual scenes, but can not give you a sense what it means to be in one because imagery is confined to the finite-size 2D screen on computers or phones - so these displays can't give you the experience what it means to have a virtual object on your table. 
        \item \todo{Brooks was of the opinion that you can measure presence through biological markers such as heart-rate. Was this done for comptuer games? Is it always necessary to scare a person to get his heart-rate going and hence give him a sense of presence?}
        \item The interaction too is not natural, e.g.: on a computer, our interactions with applications is through indirect tools such as keyboards, mice, styluses, and other peripherals. The situation is better on a mobile phone where we can actually touch things and use haptic feedback. However, some things like looking at a 3D model or CAD modelling are quite difficult on both computers and mobile phones, e.g., in a computer, one needs a constant flow of imagination to get anything done, and interacting with 3D models and CAD models on mobile phones is beyond difficult.
        \item Historically, this problem has been attempted to solve through a number of directions, some of which are digressions from the real-goal \todo{Define the real goal}: Spatial Augmented Reality, and Augmented Reality.
        \end{compact_todolist}
    \item What is the difference between Virtual Reality and Augmented Reality from a display's perspective?
        \begin{compact_todolist}
        \item Virtual Reality dispays are significantly easier to build. Why?
        \item Similarities of Virtual and Augmented Reality: 
            \begin{compact_todolist}
            \item Both need systems composed of similar technologies: Rendering, Tracking, Display
            \end{compact_todolist}
        \item Dis-similarities of Virtual and Augmented Reality: 
            \begin{compact_todolist}
            \item Virtual Reality systems are much easier to build. Why? Lower design constraints on display. Lower constraints on latency. No requirement of occlusion. Lower constraints on dynamic range. No requirement for 3D reconstruction of environment.
            \item One unique and additional requirement of VR systems is the need to reconstruct one's own body and display it within the headset.
            \item View of the real-world should not be obstructed for Augmented Reality displays
            \item VR tries to completely remove the user from the real world and immerse them in a virtual world, where as AR augmentes the real-world with virtual objects. As such, in most scenarios, the virtual objects need to obey the physical properties of the 3D environment in which they are displayed, e.g. virtual objects typically should not . But of course, these rules can be broken for example to show what is on the other side of the wall. 
            \end{compact_todolist}
        \item Hybrid systems: Video-See-Through
        \end{compact_todolist}
    \item Q: Ok, I now understand what Augmented Reality Displays are. Now what? A: Now, you need to understand limitations of current generation AR displays, and the focus of this dissertation.
    \item Q: Ok, I undertand your thesis work. Now, in what ways can your thesis be used for future research?
    \item Potential of seamlessly combining physical and digital content
    \item Requires displays to support all depth cues
    \item Do not support accommodation, defocus blur, occlusion
    \item Explanation for accommodation
    \item \todo{Explanation for defocus blur}
    \item Explanation for occlusion
    \item \todo{What other depth cues do we know of that this dissertation does not cover}
    \item What is the impact of solving these problems?
    \end{compact_itemize}
\item Background
    \begin{compact_itemize}
    \item 
    \end{compact_itemize}
\item Volumetric Displays
    \begin{compact_itemize}
    \item 
    \end{compact_itemize}
\item Varifocal-Occlusion Display
    \begin{compact_itemize}
    \item 
    \end{compact_itemize}
\item Future Work
    \begin{compact_itemize}
    \item \todo{Perceptual studies}
    \item \todo{Joint photorealistic augmentation}
    \item \todo{Focus-tunable optics based on liquid crystal lenses}
    \end{compact_itemize}
\end{compact_itemize}

\section{Notes from PhD progress update}
\begin{itemize}
\item MS: You have some knobs (number of depth planes, color bit-depth, LED bit-depth, Lens function, DMD framerate, decomposition algorithm, depth cues, time to render and decompose); I want to see a design space analysis, trade-offs for the various parameters in the display
\item MS: Discuss various possible approaches for scene decomposition, various possibilities choices in the optimization regime. Solve for one particular case.
\item KR: Discuss how various other display technologies can be emulated by your display
\item KR: Compare your display against other similar recent displays. Can we prove that our display is better?
\item DL,GW: Not clear what you are optimizing. Is it the depth blur? Are there other things that you could optimize for? We need to see your current optimization effort and future ones written in a much more concrete and formalized manner. 
\item LM: Discuss artifacts that may be possible in a real-time display because the different depth planes have different update rates (e.g. planes in the middle have equally spaced update rates whereas planes at the far or near have different updates - one very short and one very long)
\item GW: Need to show convergence analysis to convince folks that your algorithm actually works
\item GW: Your current approach is called projected-gradient approach
\item DL,LM: Look into error diffusion. This is a known technique in computer graphics. What you need is something like 3D error diffusion or 3D dithering.
\item LM: You can redo a lot of perceptual experiment with this display. You're holding yourself back by not doing them. Such studies would be real contribution to science rather than engineering which is what most of your work is about.
\item TW, MS: Please share with us a block diagram of your real-time system
\end{itemize}

