\section{Discussion}
\label{sec:volumetric:discussion}

\subsection{Limitations}
\paragraph{Bulky Optics}
The bulk of the optics is due to the large optical engine of the DLP Discovery 4100 kit, and the tiny aperture of the focus-tunable lens. Other DMD development boards have much smaller optics, and we also note that there is a commercially available AR display that uses a DMD chip~\cite{Dewald2016Avegant}. The small aperture of the focus-tunable lens constrains the optical design and limits the etendue of the system. There are focus-tunable lenses with a wider aperture that could be used. Our NED, if implemented with alternative components, could approach moderate form factor.

\paragraph{Bulky electronic components}
All of the driving electronics (DLP Discovery 4100 kit, custom RGB LED controller, microcontroller) could be reimplemented in a compact ASICs device. 

\subsection{Future Work}
Our near-eye display can emulate some other display technologies, such as multifocal and varifocal displays, and is thus suitable as a versatile platform for user studies. 
The current work could benefit greatly from a compact, wearable, wide-FoV, binocular, and real-time implementation. 
Since the hardware platform and application are similar, this work could be integrated with recent low-latency~\cite{Lincoln2016motion}, and HDR AR~\cite{Lincoln2017scene} displays work. 
This would require a holistic approach to a near-eye display rendering pipeline. 
Another opportunity for research is to investigate if this display can be made entirely independent of eye-tracking requirements. 
Another avenue for future work is to explore adaptive lens functions. 
While this dissertation always oscillates the focal length of the lens according to a sinusoidal or triangular waveform at 60 Hz, the lens is capable of following any arbitrary current waveform. 
While our display demonstrates 280 dioptrically equidistant depth planes, an adaptive lens function can give an adaptive depth distribution of depth planes. 
Uses of such adaptive depth distribution may be foveation in depth, getting high-quality perceptual performance while using fewer depth planes, etc. 

\section{Conclusion}
We have introduced a near-eye volumetric display capable of presenting a large volume over an extended depth-of-field created external to the display's physical volume. We view our system as a hybrid between traditional volumetric displays that create the volume within the confines of the display's physical volume, and view-dependent multifocal near-eye displays. We presented the optical design of our implementation and the rendering pipeline that synthesizes the volume for our display. 
Our main contribution is the idea that color-to-binary volume decomposition can be performed on a voxel-basis rather than an image-basis. 
We propose multiple decomposition algorithms and compare them with each other. 
We demonstrate a static display system which shows full-color volumetric display refreshed at 60 Hz and comprising 280 focal planes, each at a unique depth, ranging from 15cm (6.7 diopters) to 4M (0.25 diopters). 
We also demonstrate a dynamic volumetric display system with 8 depth planes. 
One of the key advantages of the proposed volumetric display system is the flexibility of the display system itself. 
It is composed of several components, each of which could be implemented and integrated in different combinations and methods to achieve different results. 
We hope that this system will inspire future research work in near-eye displays to rethink the rendering pipeline. 

