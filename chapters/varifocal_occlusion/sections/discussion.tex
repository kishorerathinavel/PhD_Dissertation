In summary, we introduce varifocal-occlusion capable AR displays based on focus-tunable optics. This approach improves the realism of optical see-through displays by enabling mutually consistent occlusions between digital and physical objects over a large depth range. We derive a formal optimization approach and real-time heuristics to tune the optical settings of our system to avoid distortions of the physical scene and demonstrate improved realism with a prototype AR display. 

\subsection{Limitations}

Similar to other varifocal-type displays, ours would require eye-tracking to determine where to focus the display. Our current prototype does not include an eye tracker, although this capability has been demonstrated with previous varifocal VR displays~\cite{Padmanaban2016Optimizing}. The field of view of our prototype is limited by the size of commercially available focus-tunable lenses, although these are steadily increasing~\cite{Padmanaban2019Autofocals}. Finally, our prototype shares limitations of other, fixed-focus occlusion-capable AR displays in being implemented as a benchtop system. Although it does not seem straightforward how to miniaturize the proposed optical design, we believe that the capabilities offered by our system are unique and important; we hope to inspire others to address some of the remaining questions on optimizing device form factors for occlusion-capable displays in general. 

\subsection{Future Work}

First and foremost, the device form factor of this and other occlusion-capable displays should be reduced to enable wearable occlusion-capable displays. This is a major optical design challenge beyond the scope of this chapter. Eye-tracking should be incorporated into such a wearable system. While most occlusion-capable displays aim at computing a binary occlusion mask, one could also envision the attenuation pattern to be optimized to enable consistent illumination, shading, and shadows of digital and physical objects along with consistent occlusion~\cite{bimber2003consistent} or enable other types of optical image processing capabilities~\cite{Wetzstein2010}. 

\subsection{Conclusion}

To enable seamless experiences with AR displays, hard-edge occlusion control is critical. With this work, we take steps towards improving the realism of optical see-through displays with varifocal occlusion capabilities. Yet, many challenges in this area remain to design and build small, light-weight AR glasses that offer perceptually realistic and seamless experiences.


%We introduced the concept of depth-dependent occlusion and presented an optimization-based approach to design dynamically changing asymmetrical optical designs which would enable varifocal occlusion. Under some reasonable constraints, we derived closed-form solutions for dynamically changing asymmetrical optical designs which would able varifocal occlusion. We built a prototype varifocal occlusion-capable AR display and compared the see-through view to previous AR displays. We hope that this work leads to more work on occlusion, mainly towards the realization of compact occlusion technologies which preserve a high fidelity of the see-through image, and towards depth-dependent occlusion.
