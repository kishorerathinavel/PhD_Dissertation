\renewcommand{\arraystretch}{1.2}
\begin{table}[!htpb]
    \begin{center}
\begin{tabular}{|>{\centering\arraybackslash}m{8cm}|>{\centering\arraybackslash}m{2.0cm}|>{\centering\arraybackslash}m{2.5cm}|}
\hline
Products/Prototypes & AR focus mechanism & Occlusion focus mechanism \\
\hline
HoloLens, Meta2, MagicLeap, etc. & Fixed-focus & None \\
\hline
\citet{Itoh2017} & Fixed-focus & Soft-edge \\
\hline
\citet{Kiyokawa2003}, \citet{Howlett2017}, \citet{Cakmakci2004} & Fixed-focus & Fixed-focus \\
\hline
~\citet{Dunn2017Wide}, \citet{Aksit2017Near} & Varifocal & None \\
\hline
\textbf{\citet{Hamasaki2019}, This chapter} & \textbf{Varifocal} & \textbf{Varifocal} \\
\hline
\end{tabular}
    \end{center}
\caption[Varifocal-Occlusion NED: comparison of focus mechanisms for virtual imagery and occlusion mask in AR displays]{Summary of the type of focus cues that are supported for the virtual imagery and for occlusion by current AR products, previous research prototypes, and this work.}
\label{tab:comparison}
\end{table}
