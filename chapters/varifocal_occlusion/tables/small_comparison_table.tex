\renewcommand{\arraystretch}{1.2}
\begin{table}
\resizebox{\columnwidth}{!}{
\begin{tabular}{|>{\centering\arraybackslash}m{5cm}|>{\centering\arraybackslash}m{2.0cm}|>{\centering\arraybackslash}m{2.5cm}|}
\hline
Products/Prototypes & AR focus mechanism & Occlusion focus mechanism \\
\hline
HoloLens, Meta2, MagicLeap, etc. & Fixed-focus & None \\
\hline
Itoh et al.\cite{Itoh2017} & Fixed-focus & Soft-edge \\
\hline
ELMO~\cite{Kiyokawa2003}, Howlett and Smithwick~\cite{Howlett2017}, Cakmakci et al.~\cite{Cakmakci2004} & Fixed-focus & Fixed-focus \\
\hline
Dunn et al.~\cite{Dunn2017Wide}, Aksit et al.~\cite{Aksit2017Near} & Varifocal & None \\
\hline
\textbf{Hamasaki and Itoh~\cite{Hamasaki2019}, This work} & \textbf{Varifocal} & \textbf{Varifocal} \\
\hline
\end{tabular}}
\caption[Varifocal-Occlusion NED: comparison of focus mechanisms for virtual imagery and occlusion mask in AR displays]{Summary of the type of focus cues that are supported for the virtual imagery and for occlusion by current AR products, previous research prototypes, and this work.}
\label{tab:comparison}
\end{table}
