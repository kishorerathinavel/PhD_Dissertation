%The word Abstract should be centered 2? below the top of the page. 
%Skip one line, then center your name followed by the title of the 
%thesis/dissertation. Use as many lines as necessary. Centered below the 
%title include the phrase, in parentheses, (Under the direction of  
%_________) and include the name(s) of the dissertation advisor(s).
%Skip one line and begin the content of the abstract. It should be 
%double-spaced and conform to margin guidelines. An abstract should not 
%exceed 150 words for a thesis and 350 words for a dissertation. The 
%latter is a requirement of both the Graduate School and UMI's 
%Dissertation Abstracts International.
%Because your dissertation abstract will be published, please prepare and 
%proofread it carefully. Print all symbols and foreign words clearly and 
%accurately to avoid errors or delays. Make sure that the title given at 
%the top of the abstract has the same wording as the title shown on your 
%title page. Avoid mathematical formulas, diagrams, and other 
%illustrative materials, and only offer the briefest possible description 
%of your thesis/dissertation and a concise summary of its conclusions. Do 
%not include lengthy explanations and opinions.
%The abstract should bear the lower case Roman number ii (if you did not 
%include a copyright page) or iii (if you include a copyright page). 

\begin{center}
\vspace*{52pt}
{\Large \textbf{ABSTRACT}}
\vspace{11pt}

\begin{singlespace}
Kishore Rathinavel: Volumetric and Varifocal-Occlusion Near-Eye Displays\\
(Under the direction of Henry Fuchs)
\end{singlespace}
\end{center}

Augmented Reality near-eye displays are a next-generation computing platform that offer unprecedented user experiences by seamlessly combining physical and digital content. However, providing a seamless and perceptually realistic experience requires displays capable of presenting photorealistic imagery, and especially, perceptually realistic depth cues, resulting in virtual imagery being presented at any depth and of any opacity. Today's commercial augmented reality displays are far from perceptually realistic because they do not support important depth cues such as accommodation and mutual occlusion, resulting in a transparent image overlaid onto the real-world at a fixed depth. Previous research prototypes fall short by presenting occlusion only for a fixed depth, and by presenting accommodation only for a narrow depth-range, or with poor depth or spatial resolution. This thesis explores a computational display approach, which is the co-design of the optics and the rendering pipeline, to address these challenges.  The approach for providing realistic accommodation is to synchronize a high-speed Digital Micro-Mirror (DMD) projector and a focus-tunable lens to periodically sweep out a volume of single-color binary images in-front of the user's eye. A novel voxel-oriented decomposition algorithm, and per-depth-plane illumination control enable full-color imagery presented over 280 depth planes refreshed at 60fps across a depth range of 15 - 400 cm. In a separate design, the approach for providing depth-dependent mutual occlusion is to design an optical system composed of focus-tunable lenses to realize a varifocal occlusion display capable of presenting imagery over a depth range of 30 - 300 cm.  Contributions of this thesis include new optical designs, new real-time rendering algorithms, and prototype displays that demonstrate accommodation and mutual occlusion depth cues over an extended depth-range. 

\clearpage

